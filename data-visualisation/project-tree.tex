\documentclass[crop,tikz]{standalone}

\usepackage{fontspec}
\setmainfont{Courier New}

\usepackage{polyglossia}
\setdefaultlanguage[spelling=modern]{russian}
\newfontfamily{\cyrillicfont}{Courier New}
\newfontfamily{\cyrillicfontsf}{Courier New}
\newfontfamily{\cyrillicfonttt}{Courier New}

\usetikzlibrary{calc,arrows.meta}

\definecolor{darkgray}{RGB}{85,85,85}

\newcounter{treeline}

\newcommand{\treeroot}[1]{ % Title
  \node[above] at (0,1) {\texttt{#1}};

  \setcounter{treeline}{0}
}

\newcommand{\treeentry}[2]{ % Title, Level
  \draw[thick]
    (#2 - 1, -\value{treeline} * 2 + 1) --
    (#2 - 1, -\value{treeline} - \value{treeline} - 1) --
    (#2 + 1, -\value{treeline} - \value{treeline} - 1) node[right] {\texttt{#1}};

  \node at (#2 + 1, -\value{treeline} - \value{treeline} - 1) [yshift=-3pt] {\textbullet};

  \foreach \x in {1,...,#2}
  {
    \draw[thick]
      (\x - 1, -\value{treeline} * 2 + 1) --
      (\x - 1, -\value{treeline} - \value{treeline} - 1);
  }

  \stepcounter{treeline}
}

\newcommand{\treeentrycommented}[4]{ % Title, Level, Dummy node, Comment
  \draw[thick]
    (#2 - 1, -\value{treeline} * 2 + 1) --
    (#2 - 1, -\value{treeline} - \value{treeline} - 1) --
    (#2 + 1, -\value{treeline} - \value{treeline} - 1) node[right] (#3) {\texttt{#1}};

  \node at (#2 + 1, -\value{treeline} - \value{treeline} - 1) [yshift=-3pt] {\textbullet};
  
  \draw[
    gray,
    line width=3pt, line cap=round, dash pattern=on 0pt off 10pt
  ] (#3) -- (20, -\value{treeline} - \value{treeline} - 1) node[darkgray,right] {#4};

  \foreach \x in {1,...,#2}
  {
    \draw[thick]
      (\x - 1, -\value{treeline} * 2 + 1) --
      (\x - 1, -\value{treeline} - \value{treeline} - 1);
  }
  
  \stepcounter{treeline}
}

\begin{document}

\begin{tikzpicture}
  \fontsize{30}{36}\selectfont
  
  \treeroot{\textbf{smartcamera/}}
    \treeentry{\_\_init\_\_.py}{1}
    \treeentrycommented{\_\_main\_\_.py}{1}{main}{Точка входа}
    \treeentry{\textbf{assets/}}{1}
    \treeentrycommented{haarcascade.xml}{2}{assets}{Классификатор лиц}
    \treeentrycommented{\textbf{database/}}{1}{database}{Директория для
    хранения баз данных}
      \treeentry{encodings.pickle}{2}
      \treeentry{states.pickle}{2}
    \treeentrycommented{\textbf{dataset/}}{1}{dataset}{Изображения для обучения распознаванию}
    \treeentrycommented{\textbf{image-processing}}{1}{image-processing}{Модуль
    обнаружения/распознавания лиц}
      \treeentry{\_\_init\_\_.py}{2}
      \treeentry{detect\_face.py}{2}
      \treeentry{encode\_face.py}{2}
      \treeentry{recognition.py}{2}
    \treeentrycommented{\textbf{telegram/}}{1}{telegram}{Модуль Telegram-bot}
        \treeentry{\_\_init\_\_.py}{2}
        \treeentry{telegram\_bot.py}{2}
        \treeentry{states\_worker.py}{2}
        \treeentry{response\_storage.py}{2}
        \treeentry{send\_file.py}{2}
        \treeentry{utils.py}{2}
    \treeentrycommented{setup.py}{1}{setup}{Конфигурационный файл
    Python проекта}
    \treeentrycommented{config.py}{1}{config}{Файл настроек Python
    проекта}
\end{tikzpicture}

\end{document}
