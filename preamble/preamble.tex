\documentclass[russian,14pt,simple, hpadding=10mm]{eskdtext}

%%% Simple - убрать лишние рамки слева

\usepackage{fontspec}
\usepackage{xltxtra} % Верхние и нижние индексы
\usepackage{xunicode} % Unicode для XeTeX

\usepackage{xecyr} % Для работы eskdx

% Шрифты, xelatex
\defaultfontfeatures{Ligatures=TeX}
\newfontfamily\cyrillicfont{Times New Roman}
\setmainfont{Times New Roman} 

\usepackage{polyglossia}
\setdefaultlanguage{russian}

% Абзацы и списки
\usepackage{indentfirst}
\usepackage{float}
\usepackage{enumerate} 

\usepackage{enumitem}
\setlist[enumerate,itemize]{itemindent=0.65em,leftmargin=3.6em}

%%% Мат. формулы
\usepackage{mathtext}

% Вставка изображений
\usepackage{graphicx}
\usepackage{xcolor}
\graphicspath{{images/}}

% Гиперссылки
\usepackage{hyperref}
\usepackage{url}

% Таблицы
\usepackage{tabularx}
\usepackage{longtable}

% Для работы eskdx
\def\No{\textnumero}

% Правильный формат подрисуночных надписей
\addto\captionsrussian{\renewcommand{\figurename}{Рисунок}}

%%% Переносы
\exhyphenpenalty=10000
\doublehyphendemerits=10000
\finalhyphendemerits=5000
\hyphenpenalty=1000

\sloppy

%%% Абзацный отступ 
\parindent=12.5mm

%%% Интервал 
\linespread{1}

\makeatletter
\renewcommand*\l@section{\@dottedtocline{1}{1em}{1em}}
\def\redeflsection{\def\l@section{\@dottedtocline{1}{0em}{10em}}}
\makeatother

%%% Раздел без цифр
\newcommand{\anonsection}[1]{
  \ESKDsectAlign{section}{Center}
  \section*{{{#1}}}
  \ESKDsectAlign{section}{Left}
  \addcontentsline{toc}{section}{#1}
}

\newenvironment{itemize*}
  {\begin{itemize}
    \addtolength{\itemindent}{-0.65em}}
  {\end{itemize}}

\newcommand{\addimg}[3]{
  \begin{figure}
    \begin{center}
      \includegraphics[scale=#2]{#1}
    \end{center}
    \caption{#3}
  \end{figure}
}

\newcommand{\addimghere}[3]{
  \begin{figure}[H]
    \setlength\abovecaptionskip{10pt}
    \setlength\belowcaptionskip{0pt}
    \centerline{
      \includegraphics[scale=#2]{#1}}
    \caption{#3}
  \end{figure}
}

\newcommand{\addtwoimghere}[4] {
  \begin{figure}[H]
    \begin{center}
      \includegraphics[scale=#3]{#1}
      \hfill
      \includegraphics[scale=#3]{#2}
    \end{center}
    \caption{#4}
  \end{figure}
}
