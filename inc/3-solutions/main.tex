\section{Обзор и анализ существующих решений}

В качестве примеров существующих на рынке СКУД, были выбраны следующие системы:

\begin{itemize*}
\item комплекcная система с распознаванием лиц Sigur;
\item Smartec ST-FR040EM;
\item комплексная система Perco.
\end{itemize*}

Важно отметить, что для выполнения сравнения системы Sigur и Perco были теоретически собраны из представленных модулей.

Краткое описание всех выбранных систем:

В продуктах Sigur распознавание лиц используется для автоматической идентификации сотрудников в точках прохода. Видеопоток может быть получен с IP-камер, подключённых к системе напрямую.

Комплексная система Sigur поддерживает как только распознавание лиц, так и другие методы идентификации совместно.

\addimghere{sigur_solution}{0.3}{СКУД Sigur}

Корректная работа гарантируется только при следующих условиях:

\begin{itemize*}
\item количество лиц в базе до 1000;
\item качественный видеопоток;
\item качественное освещение.
\end{itemize*}

При этом СКУД Sigur требует для работы компьютер/сервер со следующими техническими характеристиками: процессор Intel Core i7, не менее 8 Гб ОЗУ.

Цена продукта определения посредством лицензирования -- выбором покупателем количества камер и максимального количества сотрудников в базе. Например, минимальная конфигурация, состоящая из:

\begin{itemize*}
\item одна камера -- 7000 руб.
\item 10 лиц сотрудников в базе -- 72000 руб.
\item контроллер -- 16170 руб.
\end{itemize*}

стоит 95170 руб.

Следующая система, биометрический считыватель ST-FR040EM марки Smartec выполняет распознавание геометрии лица, а также идентификацию пользователей по коду доступа и картам стандарта Em Marine. Наличие встроенного контроллера позволяет ему выполнять функции СКУД в автономном режиме. 

\addimghere{smartec}{0.9}{СКУД Smartec ST-FR040EM}

На передней панели считывателя под небольшим углом к вертикальной плоскости расположены две камеры: обычная цветная и камера, фиксирующая изображение в ИК-диапазоне. При этом распознавание лиц выполняется с помощью обработки кадров, зафиксированных ИК-камерой.

Помимо биометрического ридера, ST-FR040EM имеет встроенный считыватель карт доступа формата Em Marine, а также оснащен экранной кодонаборной клавиатурой. Благодаря этому, устройство позволяет реализовать не только распознавание лиц, но и осуществлять контроль доступа по картам, коду, а также использовать эти методы идентификации в различных комбинациях.

ST-FR040EM имеет релейный выход с НЗ/НР контактами, через который считыватель может управлять электрическим дверным замком. НР-контакт применяется в тех случаях, когда устройство для распознавания лиц должно управлять замком, открывающимся при подаче напряжения, а НЗ – когда замок открывается при отключении питания. При этом, если рабочее напряжение замка составляет 12 В, то для него и устройства распознавания можно использовать единый источник электропитания достаточной мощности. 
 Система обеспечивает корректную работу при выполнении следующих условий:

\begin{itemize*}
\item до 500 лиц в базе распознавания;
\item температурах работы от 0 до 50 С;
\item питание не более 400 мА.
\end{itemize*}


Розничная цена СКУД Smartec составляет ~29000 руб.

Системы контроля доступа PERCo-S-20 интегрированы с биометрическими контроллерами Suprema предназначенными для учета отпечатков пальцев сотрудников и посетителей. В качестве идентификаторов в системе могут использоваться карты доступа и отпечатки пальцев совместно или по отдельности.

\addimghere{perco}{0.6}{биометрический СКУД Perco Suprema}

Для функционирования оборудования Suprema необходимо приобрести, как минимум, Базовое ПО PERCo-SN01 и один контроллер PERCo с интерфейсом связи по Ethernet.

Система поддерживает хранение до 10 отпечатков для одного сотрудника/посетителя, а максимальное количество отпечатков, хранимых в СКУД -- 20000.

При использовании контроллеров и считывателей Suprema могут быть использованы как недельные, так и сменные графики доступа для сотрудников.

Интеграция настольных считывателей серии BioMini позволяет регистрировать биометрические данные сотрудников/посетителей централизованно, например, сотрудником отдела кадра или бюро пропусков.

Цена подобной системы составляет 22470 руб.
