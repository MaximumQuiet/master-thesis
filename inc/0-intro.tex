\anonsection{Введение}

Система контроля и управления доступом (далее СКУД) - это совокупность
программных и аппаратных средств, предоставлюящая возможность управления пропускным режимом, с целью ограничить доступ к определённым территориям или помещениям лицам, не имеющим к ним разрешения. Подобные системы широко используются среди не только крупных организаций и предприятий, но и малого бизнеса, индивидуальных предпринимателей. Причинами такого успеха являются:

\begin{itemize*}
\item централизованное управление пропускным режимом на объекты;
\item сокращение времени на проверку документов;
\item упрощение ведения статистики.
\end{itemize*}

Для обеспечения контроля доступа в больших предприятий существует большое количество решений на рынке, однако для малых помещений с соответсвенно пониженной ценой наблюдается недостаток предложений. Именно поэтому, до сегоднешнего дня, малые организации используют простые замки или, к примеру, домофоны. Такие устройства значительно снижают удобство и быстроту доступа к определенным объектам. 

Решением данной проблемы является разработка собственной СКУД, что определяет цель дипломного проекта. Для достижения данной цели должны быть решены следующие задачи:

\begin{itemize*}
\item изучение теории по системам СКУД;
\item анализ существующих решений на рынке, выявление их основных возможностей и недостатков;
\item изучение теории по методам идентификации пользователей;
\item выбор программных и аппаратных инструментов для разработки системы СКУД.
\end{itemize*}
