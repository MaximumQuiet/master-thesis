\subsection{Подготовка и настройка платформы СКУД}

Аппаратной основой системы контроля и управления доступом является мини-компьютер Raspberry Pi 3B. Для обеспечения работы всех аппаратных и программных модулей СКУД требуется произвести выбор и установку дистрибутива операционной системы, её конфигурирование, настройку разъёмов, проводных и беспроводных подключений.

\subsubsection{Выбор и установка операционной системы}

Выбор операционной системы в первую очередь зависит от характеристик системы. Raspberry Pi 3B имеет процессор с ARM-архитектурой (Advanced RISC Machine) ARMv7. Такие процессоры часто требуют меньшее количество транзисторов, чем с архитектурой CISC, что позволяет уменьшить цену, потребление электроэнергии и тепловыделение. Характеристики процессоров  ARM отлично подходят для легких, портативных устройств, например смартфонов, нетбуков, планшетов. 

Архитектура ARM поддерживает 64-битный набор инструкций начиная с ARMv8, поэтому для установки на Raspberry Pi требуется операционная система, поддерживающая 32-битные ARM процессоры.

Под критерии подходят довольно большое количество операционных систем, так как сообщество Raspberry Pi активно их разрабатывает под самые различные задачи, например Open Source Media Center -- предназначен для создания домашнего киноцентра, или Pi MusicBox -- основан на Mopidy Music Steraming Server, используется для проигрывания музыки из Spotify, Google Music и т.п.

Но большинство из таких операционных систем предназначены для решения конкретной задачи и имеют предустановленные тематические приложения. Для разработки СКУД требуется операционная система имеющая инструменты для работы с контактами GPIO, поддерживающая все необходимые для разработки программные пакеты.

Также операционные системы подходящие для Raspberry Pi делятся на несколько групп: 
\begin{itemize*}
\item Unix-like системы (основанные на Linux, BSD и пр.);
\item разрабатываемые компанией Microsoft (Microsoft Windows);
\item разрабатываемые компанией Google Chrome OS и Android.
\end{itemize*}

Для выполнения проекта была выбрана операционная система, разрабатываемая непосредственно для Raspberry Pi -- Raspbian. Raspbian -- это операционная система на основе ядра Linux, являющаяся потомком системы Debian. Данный дистрибутив включает в себя множество полезных инструментов для разработки -- интерпретатор Python, редактор кода, пакеты для работы с камерой и контактами GPIO. 

Установка дистрибутива операционной системы тривиальна -- требуется записать ISO образ дистрибутива Raspbian на карту памяти и загрузить операционную систему.

%\subsubsection{Подготовка операционной системы для разработки}


