\subsection{Аппаратная часть СКУД}

Разработка системы контроля и управления доступом осуществляется на платформе
компании Raspberry Pi Foundations -- Raspberry Pi 3B \cite{pi-hardware}. Основой данного продукта является процессор с ARM-архитектурой Cortex-A53 с частотой  1,2 ГГц и модуль оперативной памяти на 1 Гб. Raspberry Pi 3B разработана с интерфейсами Ethernet и USB, что позволяет использовать многочисленные устройства расширения. 

Поддержка технологий Wi-Fi и Bluetooth обеспечивает широкие возможности для организации соединения и контактирования внешних устройств с платой.

Raspberry Pi имеет контакты GPIO, что может быть использовано для программного управления различными устройствами.

Выбранное устройство имеет малые габариты -- 85,6×53,98×17 мм. Это подразумевает то, что конечное устройство будет занимать минимальное количество места. % TODO: убрать устройства

\addimghere{raspberry-pi-board}{1}{Raspberry Pi 3B}

Raspberry Pi имеет слот CSI, представляющий интерфейс между платой и модулем камеры. Такая возможность позволяет передавать данные с камеры со скоростью до 5 Гбит/с и не использовать IP-камеры, имеющие задержку из-за передачи данных по сети.

Для создания видеопотока для обработки следует использовать камеру. Благодаря описанному разъёму CSI плата может использовать модуль камеры с соотвествующим интерфейсом.

В качестве модуля камеры был выбран Raspberry Pi Camera Board v2.1 -- модуль от производителя основной платы. Это обеспечивает совместимость камеры с компьютером, а также даёт возможность использовать официальный SDK для работы с камерой, тем самым получая наилучшее быстродействие.

\addimghere{camera-board}{0.3}{модуль камеры Raspberry Pi Camera Board}

Модуль камеры основан на сенсоре Sony IMX 219 PQ, имеет разрешение до 8 Мп (3280х2464) и поддерживает видеоформаты от 480p (90 FPS) до 1080p (30 FPS).

Raspberry Pi не имеет стандартных разъёмов для подключения жестких дисков, но содержит слот для SD-карты. Такое решение было сделано для большей компактности. Для в качестве хранилища информации была выбрана SD-карта {название}. %TODO: название карты памяти, описать хар-ки.

Общая архитектура аппаратной части системы контроля и управления доступом:

\addimghere{hardware-diagram}{0.8}{структура аппаратной части системы}

