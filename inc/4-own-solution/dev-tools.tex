\subsection{Выбор инструментов разработки}

В рамках разработки системы контроля и управления доступом выбор инструментов зависит не только от возможностей тех или иных библиотек конкретных языков программирования, но и от задач поставленных в разделе 3.1.

В результате проведения обзора высокоуровневых языков, в качестве которых были выбраны C++, Java, Python, был сделан выбор в пользу Python по следующим причинам:

\begin{itemize*}
\item код программы выполняется интерпретатором, что даёт возможность запускать программу без процесса компиляции. Это ускоряет процесс разработки и отладки программного обеспечения;
\item интерпретатор Python занимает гораздо меньший объём памяти по сравнению с JRE;
\item большая кроссплатформенность в отличие от C++;
\end{itemize*}

\subsubsection{Работа с камерой}

В качестве решения для работы с модулем камеры была выбрана официальная библиотека для Raspberry Pi Camera Board -- picamera. Она предоставлет интерфейс для камеры на языке программирования Python.

picamera позволяет выполнять такие операции как:

\begin{itemize*}
\item работа с изображением с камеры, сохранение, обработка (поворот, изменение цвета и т.п.);
\item получение видеопотока, возможность выбора формата и количества кадров в секунду.
\end{itemize*}

\subsubsection{Работа с распознаванием лиц}

Для распознавания лиц был выбран инструмент Face Recognition, который представляет из себя библиотеку, имеющюю интерфейс на языке Pythion. Данная библиотека использует два основных инструмента:

\begin{itemize*}
\item библиотеку OpenCV -- широко распространённый набор алгоритмов компьютерного зрения, обработки изображений и численных алгоритмов. Включает в себя базовые структуры, вычисления (математические функции, генераторы случайных чисел), обработку изображений, модели машинного обучения, модули для работы с калибровкой камеры.
\item библиотеку dlib, для работы с нейронными сетями, машинным обучением и пр. Использует обученные каскады для поиска лиц.
\end{itemize*}

Используя возможности детектирования лиц и обученных нейросетей Face Recognition добивается максимального процента распознавания лиц.
