\subsection{Выбор инструментов разработки}

В рамках разработки системы контроля и управления доступом выбор инструментов зависит не только от возможностей тех или иных библиотек конкретных языков программирования, но и от задач поставленных в разделе 3.1.

В результате проведения обзора высокоуровневых языков, в качестве которых были выбраны C++, Java, Python, был сделан выбор в пользу Python по следующим причинам:

\begin{itemize*}
\item код программы выполняется интерпретатором, что даёт возможность запускать программу без процесса компиляции. Это ускоряет процесс разработки и отладки программного обеспечения;
\item интерпретатор Python занимает гораздо меньший объём памяти по сравнению с JRE;
\item большая кроссплатформенность в отличие от C++;
\end{itemize*}

\subsubsection{Работа с камерой}

В качестве решения для работы с модулем камеры была выбрана официальная библиотека для Raspberry Pi Camera Board -- picamera. Она предоставлет интерфейс для камеры на языке программирования Python.

picamera позволяет выполнять такие операции как:

\begin{itemize*}
\item работа с изображением с камеры, сохранение, обработка (поворот, изменение цвета и т.п.);
\item получение видеопотока, возможность выбора формата и количества кадров в секунду.
\end{itemize*}

\subsubsection{Работа с распознаванием лиц}

Для распознавания лиц был выбран инструмент Face Recognition, который представляет из себя библиотеку, имеющюю интерфейс на языке Python. Данная библиотека использует два основных инструмента:

\begin{itemize*}
\item библиотеку OpenCV -- широко распространённый набор алгоритмов компьютерного зрения, обработки изображений и численных алгоритмов. Включает в себя базовые структуры, вычисления (математические функции, генераторы случайных чисел), обработку изображений, модели машинного обучения, модули для работы с калибровкой камеры.
\item библиотеку dlib, для работы с нейронными сетями, машинным обучением и пр. Использует обученные каскады для поиска лиц.
\end{itemize*}

Используя возможности детектирования лиц и обученных нейросетей Face Recognition добивается максимального процента распознавания лиц.

\subsubsection{Построение чат-бота}
Для разработки чат-бота в мессенджере Telegram была использована имплементация API Telegram на языке Python -- pyTelegramBotAPI. Одна из ключевых возможностей инструмента -- использование паттерна проектирования Decorator, предназначенного динамического подключения дополнительного поведения к объекту.

Основные возможности pyTelegramBotAPI:

\begin{itemize*}
\item получение, обработка и отправка нескольких видов сообщений и медиафайлов -- текстовых сообщений, аудио, видео, документов различных форматов, голосовых сообщений, стикеров и т.д.;
\item управлять сообщениями, т.е. отправлять, получать, удалять, редактировать;
\item генерировать меню и inline-меню (Рисунок 16)
\item использовать механизм web-hooks;
\item использовать любые прокси-серверы;
\item собирать отчёты о работе API.
\end{itemize*}

\addtwoimghere{telegram-menu}{telegram-inline-menu}{0.28}{пример меню (слева) и inline-меню (справа)}

\subsubsection{Выводы по разделу 3}

Проведён обзор платформы для разработки СКУД. Были выявлены основные требования к аппаратным и программным модулям системы. В результате были спроектированы аппаратная и программная части системы, перечислены основные используемые устройства и библиотеки.

\addimghere{detailed-structure}{0.6}{Уточнённая структура программного обеспечения СКУД}
