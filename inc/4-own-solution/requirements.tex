\subsection{Требования и задачи}

Для создания системы контроля и управления доступом были сформулированы наиболее важные идеи, задачи и требования на основе проведённого теоретического анализа и обзора существующих решений.

Разрабатываемая система должна:

\begin{itemize*}
\item осуществлять контроль доступа на основе биометрического метода идентификации -- распознавания лиц. Такой метод является наиболее практичным для небольшого количества сотрудников, т.к. позволяет не приобретать карты доступа, а также даёт возможность более быстрой идентификации.
\item иметь реле для управления различными видами замков, турникетов, и т.п.
\item иметь клиентскую часть, представленную в виде чат-бота в мессенджере. Исследование рынка показало, что методы управления СКУД неудобны, не имеют большого функционала и часто требуют непосредственного нахождения рядом с управляющим блоком. Основные модули, из которых должна состоять клиентская часть: модуль обработки видеопотока и распознавания лиц; модуль чат-бота мессенджера. В качестве мессенджера был выбран Telegram, так как имеет самое функциональное и документированное API для создания чат-ботов и сами чат-боты Telegram отличаются от всех мессенджеров наличием таких особенностей как: возможность общения с ботом посредством клавиатур, команд (например, /help или /menu), обычных сообщений и картинок; возможность построения многоуровневых меню, что решает проблему незнания пользователем команд управления; модуль управления реле. Модуль отвечает за принятие команд от модуля распознавания, подачей напряжения на контакты GPIO для управления реле.
\item обеспечивать безопасность хранения информации.
\end{itemize*}
