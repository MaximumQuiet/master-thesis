\subsection{Программное обеспечение СКУД}

Разработка программного обеспечения СКУД осуществляется на основе трёх модулей: чат-бота, модуля распознавания лиц и модуля управлдения реле.

Основная схема программного обеспечения:

\addimghere{software-diagram}{1}{структура программной части СКУД}

\subsubsection{Чат-бот}

Задача данного программного модуля состоит в получении пользовательских команд, обработки, конструирования и отправки ответов.

Основная функциональность, выполняемая модулем:

\begin{itemize*}
\item построение пользовательских меню;
\item построение диалогов с пользователем;
\item проведение аутентификации пользоваетеля чат-бота;
\item управление базой лиц: добавление, удаление, модификация;
\item уведомеления о происходящих событиях пользователя (например проход распознанного/нераспознанного лица).
\end{itemize*}

\subsubsection{Модуль распознавания лиц}

Модуль распознавания лиц выполняет одну из главных задач системы контроля и управления доступом -- идентификацию пользователей. Функциональность модуля:

\begin{itemize*}
\item обработка и подготовка изображения для детектирования лиц;
\item детектирование лица на фотографии, составление выделющих масок;
\item распознавание лиц, идентификация лиц с заранее заданными фотографиями пользователей в режиме единичной фотографии или в режиме потока кадров;
\item принятие решения об подаче сигнала открытия/закрытия на реле.
\end{itemize*}

\subsubsection{Модуль управления реле}

Данный модуль представляет простое сокрытие реализации работы с контактами GPIO, то есть предоставляет интерфейс для управления состояниями реле.

Функциональность данного модуля состоит из:

\begin{itemize*}
\item подачи сигнала отключения реле;
\item подачи сигнала включения реле. 
\end{itemize*}

