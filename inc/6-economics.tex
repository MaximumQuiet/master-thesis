\section{Экономическая часть}

Рассчитаем общую стоимость разработки проекта по формуле \ref{cost}:

\begin{equation}\label{cost}
\textup{З} = \textup{З}_\textup{к} + \textup{З}_\textup{зп},
\end{equation}

где З -- единовременные затраты, руб.,
З\uscript{к} -- затраты на комплектующие, руб., 
З\uscript{зп} -- затраты на заработную плату исполнителей, руб. 

Для сборки системы контроля и управления доступом потребовались комплектующие,
расчёт затрат представлен в таблице \ref{acessories-cost}:

\begin{table}[H]
  \caption{Комплектующие системы контроля и управления доступом}\label{acessories-cost}
  \centering
  \begin{tabularx}{\textwidth}{|X|l|l|X|}
  %\begin{tabular}{| >{\centering}p{3cm} | >{\centering}p{9cm} | >{\centering}p{1.4cm} | >{\centering}p{3cm} |}
  \hline № & Наименование & Кол-во, шт. & Цена, руб.
  \tabularnewline
  \hline 1 & Мини-компьютер Raspberry Pi 3B & 1 & 2284 
  \tabularnewline
  \hline 2 & Камера Raspberry Camera Board & 1 & 850 
  \tabularnewline
  \hline 3 & SD-карта Microdata 8 Гб & 1 & 160 
  \tabularnewline
  \hline 4 & Корпуc Aokin & 1 & 150 
  \tabularnewline
  \hline 5 & Зарядное устройство Smartbuy SBP-6000 & 1 & 150 
  \tabularnewline
  \hline & Итого & 5 & 3594 
  \tabularnewline
  \hline 
  \end{tabularx}
\end{table}

Таким образом, затраты на комплектющие системы составляют 3594,00 рублей. Для
определения полной стоимости СКУД в таблице \ref{employees-cost} был провёден расчёт заработной платы
исполнителя:

\begin{table}[H]
  \caption{Исполнители проекта}\label{employees-cost}
  \begin{tabularx}{\textwidth}{|X|l|l|X|} 
  %\begin{tabular}{| >{\centering}p{5cm} | >{\centering}p{3.3cm}
  %| >{\centering}p{3.3cm} | >{\centering}p{3.3cm} |}
  \hline Исполнитель & Трудоёмкость, дней & Дневная ставка, руб. & Зарплата, руб. 
  \tabularnewline
  \hline Программист & 26 & 900 & 23400 
  \tabularnewline
  \hline 
  \end{tabularx}
\end{table}

Общая смета на разработку системы контроля и управления доступом представлена
в таблице \ref{counts}:

\begin{table}[H]
  \caption{Смета на разработку проекта}\label{counts}
  \begin{tabularx}{\textwidth}{|X|l|}
  %\begin{tabular}{| >{\centering}p{7.4cm} | >{\centering}p{7.4cm} |}
  \hline Статья затрат & Сумма, руб. 
  \tabularnewline
  \hline Комплектующие & 3594 
  \tabularnewline
  \hline Заработная плата & 23400 
  \tabularnewline
  \hline Итого & 26994 
  \tabularnewline
  \hline 
  \end{tabularx}
\end{table}

Итоговая стомость системы контроля и управления доступом составляет 26994,00
рубля. Ближайшие аналоги имеют стоимость от 20000,00 рублей и выше только за одного
пользователя. Из расчитанной стоимости можно сделать вывод
о конкурентоспособности проекта. Система контроля доступа обладает приемлемой
стоимостью для индивидуальных предпринимателей, имеющих небольшие помещения,
нуждающиеся в контроле доступа. Разработка данного проекта является
экономически целесообразна. 
