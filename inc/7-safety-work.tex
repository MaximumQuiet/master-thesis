\section{Охрана труда и промышленная экология}

Все интегральные микросхемы, содержащиеся в системе, подключены к источнику
питания и снабжены блокировочными конденсаторами. Напряжение питания,
необходимое для работы системы составляет 5 В.

К опасным производственным факторам можно отнести электрический ток и факторы пожара.

К вредным физическим производственным факторам, связанных с проектированием принципиальной электрической схемы на ПК с помощью программных средств, можно отнести: повышенные уровни электромагнитного излучения, статического электричества, запыленности воздуха рабочей зоны нерациональная организация освещенности рабочего места. 

К последствиям психофизиологических факторов относятся: болезни глаз из-за перенапряжения зрения, нервные заболевания из-за напряжения внимания, интеллектуальных, больших информационных нагрузок и эмоциональных перегрузок, а также монотонности труда, болезни позвоночника, спины, шеи из-за длительных статических нагрузок и нерациональной организации рабочего места. 

Эти физические и психофизиологические факторы следует исключить организационными и техническими мерами защиты, выбором программного обеспечения, соответствующего качества и характеризующегося основными принципами организации взаимодействия «человек-машина». 

Любую электроустановку следует рассматривать как возможный инициатор воспламенения. К основным факторам, приводящим к возгоранию электротехнических изделий, относятся чрезмерный нагрев электрическим током отдельных деталей и электрическая дуга. Для предупреждения пожаров необходимо соблюдение общих норм пожаробезопасности согласно ГОСТ 12.1.004-91 [5]. 5 - ГОСТ 12.1.004-91. Пожарная безопасность. М.: ИПК Издательство стандартов, 1996. 

При работе в помещении с указанным выше электрооборудованием не применяются легковоспламеняющиеся и горючие вещества. Пожарная безопасность электрооборудования обеспечивается наличием предохранителей в цепях питания, а также УЗО. Помещение оборудовано порошковым огнетушителем ОП - 3, предназначенным для тушения твердых, жидких и газообразных веществ, а также для тушения электроустановок, находящихся под напряжением до 1000 В. В целях пожарной профилактики в помещении проводятся следующие мероприятия:

\begin{itemize*}
\item производится периодический инструктаж работников, работающих с электрооборудованием, о мерах пожарной безопасности;
\item работники постоянно закрепляют навыки обращения со средствами пожаротушения;
\item средства пожаротушения располагаются в доступных местах;
\item все электроприборы проходят регулярное и своевременное техническое обслуживание. Все электрооборудование используется в строгом соответствии с инструкциями по эксплуатации. Не допускаются такие режимы работы оборудования, при которых происходит его перегрев, образование искр или электрической дуги.
\end{itemize*}
