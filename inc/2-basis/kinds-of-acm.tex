\subsection{Виды СКУД}

Все системы контроля и управления доступом можно разделить на две категории: сетевые и автономные системы.

В сетевой системе все контроллеры соединены с компьютером, что позволяет управлять десятками дверей, проходных пунтков, турникетов. Подобные системы удобны для больших объектов (офисов, производственных предпрятий).

Сетевые системы используются для:

\begin{itemize*}
\item использования сложных алгоритмов допуска сотрудников с разными привидегиями в разные зоны объекта;
\item организации учёта рабочего времени;
\item при взаимодействии с другими системами безопасности, например с пожарной сигнализацией.
\end{itemize*}

В сетевой СКУД могут применяться как проводные, так и беспроводные методы передачи данных, например:

\begin{itemize*}
\item Bluetooth;
\item Wi-Fi;
\item GSM.
\end{itemize*}

Автономные системы менее функциональны, дешевле, проще в эксплуатации. Они не требуют прокладки сотен метров кабеля, а также сопряжения и управления с компьютера. При этом, автономные системы могут иметь некоторый функционал сетевых СКУД, например, ведение отчётов, удалённое управление, но должны обеспечивать безопасность хранения информации, т.к. все идентификаторы располлагаются непосредственно в автономной системе.
