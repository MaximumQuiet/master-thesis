\subsection{Основные составляющие СКУД}

\subsubsection{Контроллер}
В современных системах контроля и управления доступом контроллер - одна из самых функциональных модулей. Он выполняет роль центрального модуля управления -- в его памяти хранятся коды идентификаторов, а значит контроллер принимает решение о допуске определённого человека на охраняемый объект.

В случае необходимости автономного контроллера, он совмещается со считывателем в одном устройстве, что позволяет сократить затраты, снизить стоимость и упростить монтаж.

При использовании СКУД на несколько точек доступа, ощутимо возрастает значимость характеристик.

\addtwoimghere{z-5r}{anviz}{0.7}{пример контролеров -- автономный Z-5R (слева), Anviz SAC844 (слева)}

\subsubsection{Считыватель}
Считыватель - устройство, которое получает код идентификатора и передаёт его в контроллер для обработки. Считыватели, в зависимости от модели, позволяют принимать в качестве идентификатора следующие идентификаторы:

\begin{itemize*}
\item proximity-карты;
\item TouchMemory;
\item код доступа;
\item биометрия (отпечатки пальцев, радужная оболочка глаз, лица).
\end{itemize*}

\addimghere{readers}{0.8}{считыватели Smartec}

\subsubsection{Программное обеспечение}
В случае использования компьютера в качестве контроллера, существует специальное програмнное обеспечение, роль которого -- управление подключёнными модулями и хранение идентификаторов.

Программное обеспечение используется при необходимости более широкого функционала, чем может предоставить обычный контроллер, например:

\begin{itemize*}
\item ведение отчётности, быстрый доступ к отчётам;
\item подключение баз данных сотрудников, например в 1C;
\item удалённое управление, изменение баз данных, конфигураций.
\end{itemize*}

\addimghere{software}{0.4}{программное обеспечение Castle}
