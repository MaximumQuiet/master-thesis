\subsection{История развития СКУД}

История СКУД начинается в связи с потребностью введения контроля над доступом людей на ограниченные территории, заменив старые, неактуальные способы контроля на автоматизированные и удобные в управлении системы. Структурная схема первых систем (на сегодняшний день также используются):

\begin{itemize*}
\item считыватель (программно-аппаратное устройство, которое принимает коды от внешниx устройств);
\item валидатор (логический блок, предназначенный для проверки кода на его соотвествие);
\item реле (программный или аппаратный модуль, предназначенный для управления устройствами блокировки прохода).
\end{itemize*}

\subsubsection{Первое поколение}

СКУД первого поколения выполняли лишь базовые функции: считыватель ключей получал определенный код, передавал его в валидатор, далее валидатор проверял код на соответствие и принимал решение о открытии/закрытии блокирующего устройства.

\addimghere{simple-skud}{0.5}{пример простого СКУД -- Z-5R}

В качестве устройств содержащих в себе ключ доступа широко использовалась TouchMemory -- устройства, имеющие однопроводный протокол обмена информацией и флеш-память для её хранения.

\addimghere{touchmemory}{0.9}{элеткронный ключ Button.com, реализующий систему TouchMemory}

Подобные устройства хоть и не отличались большой функциональностью, внесли в развитие СКУД несколько нововведений:

\begin{itemize*}
\item отказ от линий связи, что привело к использованию паллиативных механизмов программирования;
\item внедрение протокола MicroLan, на основе которого позже будет создано множество охранных и пожарных систем.
\end{itemize*}

Несмотря на все достоинства, СКУД первого поколения не отвечали еще нескольким основным требованиям -- ведению журнала событий и простоте программирования контроллеров.

\subsubsection{Второе поколение}

Опыт использования первых СКУД показал, что они востребованы, однако пользователи нуждаются в увеличении функциональности и безопасности.

В связи с увеличением мощности микроконтроллеров, а также появлением доступа к технологиям Ethernet. Из этого исходят следующие нововведения:

\begin{itemize*}
\item появились журналы событий. Они не работали в реальном времени, но значительно повышали безопасность охраняемых объектов;
\item контроллеры стали способны сами получать и передавать данные для обработки, без необходимости в управляющих командах;
\item на платах СКУД предусматривался канал связи Ethernet. Отныне локальные сети стали ключевыми линиями связи.
\end{itemize*}

\addimghere{skud-sigur}{0.3}{Сетевая СКУД Sigur -- поддерживает работу по сети за счет встроенного Ethernet порта}

В процессе использования СКУД нового поколения, были выявлены следующие недостатки:

\begin{itemize*}
\item журналы событий не работали в реальном времени. Такая функция позволила бы незамедлительно реагировать на определенные события;
\item не было поддержки более сложных алгоритмов валидации ключей, соответственно безопасность нельзя было повысить;
\item не поддерживались разные виды идентификаторов, такие как штрих-коды, отпечатки пальцев и т.п.
\end{itemize*}

\subsubsection{Третье поколение}

Применение локальных сетей ускорило развитие СКУД. Ранее совмещенные элементы систем контроля и управления доступом стали разделяться, образовывая распределенные системы. Это означало, что стало возможным располагать на разные объекты считыватели, подключённые к одному контроллеру СКУД.

Также, чаще роль контроллера СКУД стал выполнять персональный компьютер, с установленным управляющим программным обеспечением. Такая структура позволила:

\begin{itemize*}
\item наделить системы СКУД гибкостью, возможностью использования различного ПО, выбираемого по требуемым характеристикам;
\item использовать любые поддерживаемые методы идентификации -- от простых TouchMemory до биометрии.
\end{itemize*}

\addimghere{skud-scheme}{0.5}{структурная схема СКУД третьего поколения}

Третье поколение СКУД вследствие больших нововведений получила и ряд нерешённых проблем:

\begin{itemize*}
\item все элементы системы теперь зависели от одного-двух серверов, на которых выполнялось управляющее программное обеспечение;
\item за счёт большого числа новых методов идентификации появилось множество проблем с безопаснотстью. Такая ситуация подтолкнула производителей к использованию и разработке новых протоколов обмена информацией между устройствами.
\end{itemize*}
