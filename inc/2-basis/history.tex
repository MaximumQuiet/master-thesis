\subsection{История развития СКУД}

История СКУД начинается в связи с потребностью введения контроля над доступом людей на ограниченные территории, заменив старые, неактуальные способы контроля на автоматизированные и удобные в управлении системы. Структурная схема первых систем (на сегодняшний день также используются):

\begin{itemize*}
\item считыватель (программно-аппаратное устройство, которое принимает коды от внешниx устройств);
\item валидатор (логический блок, предназначенный для проверки кода на его соотвествие);
\item реле (программный или аппаратный модуль, предназначенный для управления устройствами блокировки прохода).
\end{itemize*}

\subsubsection{Первое поколение СКУД}

СКУД первого поколения выполняли лишь базовые функции: считыватель ключей получал определенный код, передавал его в валидатор, далее валидатор проверял код на соответствие и принимал решение о открытии/закрытии блокирующего устройства.

\addimghere{simple-skud}{0.5}{пример простого СКУД -- Z-5R}

В качестве устройств содержащих в себе ключ доступа широко использовалась TouchMemory -- устройства, имеющие однопроводный протокол обмена информацией и флеш-память для её хранения.

\addimagehere{touchmemory}{0.5}{элеткронный ключ Button.com, реализующий систему TouchMemory}

Подобные устройства хоть и не отличались большой функциональностью, внесли в развитие СКУД несколько нововведений:

\begin{itemize*}
\item отказ от линий связи, что привело к использованию паллиативных мехнизмов программирования;
\item внедрение протокола MicroLan, на основе которого позже будет создано множество охранных и пожарных систем.
\end{itemize*}

Несмотря на все достоинства, СКУД первого поколения не отвечали еще нескольким основным требованиям -- ведению журнала событий и простоте программирования контроллеров.
